\documentclass[12pt]{article}

\parindent 0.cm
\begin{document}
%\topmargin -2cm
\topmargin-1cm

\setlength{\unitlength}{1cm}
\begin{picture}(15,0)
%\put(-8.5,-15.5){\special{psfile="chouettesLPS.ps"}}
%\put(-11.,-15.5){\special{psfile="chouettesLPS.ps"}}
%\put(3.,-14.7){\special{psfile="logoENS.vert-sature-trans.ps"}}
\end{picture}

\vspace{-1cm}

\pagestyle{empty} 
%\textwidth 6.0in
%\textheight 12.0in
%\oddsidemargin -0.5 14615-20cm
\centerline{\bf  Ecole Normale Sup\'erieure}
\vskip .1cm
\centerline{\bf \large LABORATOIRE DE PHYSIQUE  STATISTIQUE}
\centerline{\bf associ\'e au C.N.R.S et aux Universit\'es Paris 6 et 7}


\centerline{24, rue Lhomond, 75231 Paris Cedex 05 }

\vskip 1cm
\begin{flushleft}
Vincent HAKIM\\
Tel: +33 1 44 32 37 68\\
e-mail : hakim@lps.ens.fr
\end{flushleft}
\vskip 1cm

Dear Editor
\vskip .5cm

we would like to submit the attached manuscript,
``{\bf Imogene : identification of motifs and cis-regulatory modules underlying
gene co-regulation}",
for publication in  Nucleic Acid Research as a {\em Standard paper} pertaining to
the  {\em Computational Biology} subject category.
\vskip .5cm

There is a strong interest in
computationally decoding the cis-regulatory information that directs gene
expression in a tissue or condition-specific manner. A common situation in practice
is the availability of a small set of  CRMs active in a condition of interest, provided either
by different classical small scale studies, or by the test of a set of CRMs identified in a large-scale study.
From these available CRMs,  one would usually  like both i) to decode the cis-regulatory information
which gives rise to expression specific to the condition of interest and ii) to obtain other CRMs specifically active in the same condition.
However, there are few existing algorithms that are able to deal with a small set of (10-20)  characterized CRMs. Moreover, the top-performing ones
are motif-blind : they address task ii) by giving up addressing task i), 
as assessed in a previous work (Kantorovitz et al, Dev. Cell 17, 568�579 (2009)), i.e. they globally characterize
the statistics of the available CRMs without providing information on transcription factor binding sites.
 
 \vskip .5cm
Our manuscript improves on this current situation. Building on our previous work ( Rouault et al, PNAS 107, 14615�20 (2010)), it
 describes  and tests {\em Imogene}, a novel algorithm 
 that is able to address both tasks i) and ii). Namely, it
determines cis-regulatory information  in multi-cellular organisms and
particularly in mammals,
 {\em de novo} from a small set of
well-characterized CRMs. It then predicts from it novel CRMs with the same specificity.
{\em Imogene} bypasses the lack of extensive data
by exploiting in a systematic way the phylogenetic information now available in the sequences
of multiple related genomes.

\vskip .5cm
We present tests of the basic capabilities of {\em Imogene}, namely creating
{\em de novo} cis-binding  motifs specific to a family of CRMs and retrieving
other CRMs with similar regulatory abilities, using available neural and limb mouse
CRMs. {\em Imogene} is found to perform on these mammalian CRMs as well as the current best motif-blind methods.
We also show that different classes of CRMs can be distinguished based on 
{\em Imogene} generated motifs. Notably, we show that only using genomic sequence information,
{\em Imogene} performs as well in this classification task as the machine learning based  on extensive ChIP-seq data presented in the noted
work of  Zinzen et al,  Nature, 462, 65�70 (2009).
\vskip .5cm

{\em Imogene} computer code is publicly available. Additionally, a web-interface has been developed
on the Pasteur Institute mobyle web platform, to render the
developed ensemble of statistical tools easily usable by the biological community.
\vskip .5cm
We believe that our results and the described software will be of 
interest to a large spectrum of readers from pure biologists to
bio-informaticians and that publication in Nucleic Acid Research is particularly
suited to reach this broad readership.
%\vskip .5cm

%We would like to suggest as members of the Editorial Board particularly
%qualified to handle our manuscript:

%G Church

%D Duboule (?)

%TR Hughes (connais pas mais peut-etre ok?)

%M Levine (peut-�tre pas de nouveau?)

%\vskip .5cm
%Qualified referees include:

%M Eisen (UC Berkeley): Computational biology and regulation of gene expression.

%G Stormo (Washington U) : Computational biology,  protein-DNA interaction.

%F Guillemot (NIMR, London, UK): pro-neural target genes in the mouse; computational prediction and functional analysis.

%R Mann (Columbia U, New-York): gene regulation in {\em Drosophila}: computational prediction and functional analysis.

\vskip .5cm
Yours sincerely,
\vskip .5cm

F Schweisguth and V Hakim 
\end{document}


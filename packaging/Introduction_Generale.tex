\newpage
\WriteThisInToc\addstarredchapter{Introduction}\chapter*{Introduction}\markboth{\slshape
Introduction\hfil}{\slshape Introduction\hfil}

\vskip7mm

\noindent\textbf{\LARGE\sffamily\scshape Les r�seaux de r�gulation
g�n�tique }

\medskip

\blstbul{15}
		
         \bs\item{\textbf{\sffamily{Bref historique}}}

            Monod, Jacob. Promoteurs. 

      \bs\item{\textbf{\sffamily{Divers modes de r�gulation}}}

      \blste{10}
		
         \item{\underline{\sffamily{Promoteurs}}}
		
		
         \medskip \item{\underline{\sffamily{Enhancers}}}
		
         \medskip \item{\underline{\sffamily{�pig�n�tique}}}
			
      \elste

\elstbul


%%%%%%%%%%%%%%%%%%%%%%%%%%%%%%%%%%%%%%%%%%%%%%%%%%%%%%%%%%%%%%%%%%%%%%%%%%%%%%%%%%%%%%%%%%%%%%%%%%%%
\noindent\textbf{\LARGE\sffamily\scshape Interactions entre les facteurs de
transcription et l'ADN.}
 
Mod�les PWM, biophysiques. Donn�es biologiques grande �chelle.


\newpage \noindent\textbf{\sffamily\scshape Quelques remarques sur la version
pdf du manuscrit}

Voici quelques remarques sur la version pdf de ce manuscrit, qui peuvent rendre
la lecture plus ais�e. Dans la table des mati�res, la liste des figures et la
liste des annexes, les titres sont des liens hypertexte qui pointent vers
l'item d�crit. Dans la liste des notations utilis�es et la bibliographie, ce
sont les num�ros de page qui sont des liens hypertexte.


\newpage
\WriteThisInToc\addstarredchapter{Introduction}\chapter*{Introduction}\markboth{\slshape Introduction\hfil}{\slshape Introduction\hfil}

\vskip7mm

\noindent\textbf{\LARGE\sffamily\scshape Puces � atomes -- contexte}

\medskip

\blstbul{15}
		
			\bs\item{\textbf{\sffamily{Bref historique}}}

  		\bs\item{\textbf{\sffamily{Divers couplages des atomes}}}

		\blste{10}
		
			\item{\underline{\sffamily{Interactions avec des fluctuations spatiales des courants de pi�geage : fragmentation}}}
		
		
			\medskip
			\item{\underline{\sffamily{Interactions avec des atomes adsorb�s sur la surface : patch-effect}}}
		
			\medskip
			\item{\underline{\sffamily{Interactions avec des photons dans une cavit� optique}}}
			
			\medskip
			\item{\underline{\sffamily{Interactions avec le bruit thermique de champ proche}}}
		
	
		\elste

\elstbul


%%%%%%%%%%%%%%%%%%%%%%%%%%%%%%%%%%%%%%%%%%%%%%%%%%%%%%%%%%%%%%%%%%%%%%%%%%%%%%%%%%%%%%%%%%%%%%%%%%%%
\noindent\textbf{\LARGE\sffamily\scshape Une puce � atomes supraconductrice  --  motivations}
 


\newpage
\noindent\textbf{\sffamily\scshape Quelques remarques sur la version pdf du manuscrit}

Voici quelques remarques sur la version pdf de ce manuscrit, qui peuvent rendre la lecture plus ais�e. Dans la table des mati�res, la liste des figures et la liste des annexes, les titres sont des liens hypertexte qui pointent vers l'item d�crit. Dans la liste des notations utilis�es et la bibliographie, ce sont les num�ros de page qui sont des liens hypertexte.


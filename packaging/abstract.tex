\begin{ThesisAbstract}

	\begin{FrenchAbstract}
	
	
			\KeyWords{R�gulation g�n�tique, Facteur de transcription, Mod�le de
         Potts, Phylog�n�tique, Algorithme bay�sien, diff�renciation musculaire, trichomes.}
	\end{FrenchAbstract}
	
   \begin{EnglishAbstract}


            Cellular differentiation and tissue specification depend in part on
            the establishment of specific transcriptional programs of gene
            expression.  These programs result from the interpretation of
            genomic regulatory information by sequence-specific transcription
            factors (TFs). Decoding this information in sequenced genomes is
            a key issue.  First, we present models that describe the
            interaction between the TFs and the DNA sequences they bind to,
            called Transcription Factor Binding Sites (TFBSs). Using a Potts
            model inspired from spin glass physics along with high-throughput
            binding data for a variety of Drosophilae and mammalian TFs, we
            show that TFBSs exhibit correlations among nucleotides and that the
            account of their contribution in the binding energy greatly
            improves the predictability of genomic TFBSs.  Then, we present an
            extension to mammalian genomes of a Bayesian, phylogeny-based
            algorithm designed to computationally identify the Cis-Regulatory
            Modules (CRMs) that control gene expression in a set of
            co-regulated genes, and that was previously applied to Drosophila
            regulation. Starting with a small number of CRMs in a reference
            species as a training set, but with no a priori knowledge of the
            factors acting in trans, the algorithm uses the over-representation
            and conservation of TFBSs among related species to predict putative
            regulatory elements along with genomic CRMs underlying
            co-regulation. We present several applications of this algorithm
            both in Drosophila and vertebrates. We also present an extension of
            the algorithm to the case of pattern recognition, showing that CRMs
            with different patterns of expression can be distinguished on the
            sole basis of their DNA motifs content.  Finally, we present
            applications of these modeling tools to real biological cases : the
            trichomes differentiation in Drosophila, and the skeletal muscle
            differentiation in the mouse. In both cases, predictions were
            experimentally validated in a joint work with biological teams, and
            point towards a great flexibility and robustness of the
            cis-regulatory processes. 
	
         \KeyWords{Gene regulation, Transcription Factor, Potts Model,
         Phylogeny, Bayesian algorithm, muscle differentiation, trichomes.}
      \end{EnglishAbstract}
	
\end{ThesisAbstract}



%%%%%%%%%%%%%%%%%%%%%%%%%%%%%%%%%%%%%%%%%%%%%%%%%%%%%%%%%%%%%%%%%%%%%%%%%%%%%%%%%%%%%%%%%%%%%%%%%%%%%%%%%%%%%%%%%%%%%%%%%%%%%%
%%%%%TOUT CE QUI CONCERNE TLORIA%%%%%%%%%%%%%%%%%%%%%%%%%%%%%%%%%%%%%%%%%%%%%%%%%%%%%%%%%%%%%%%%%%%%%
%%%%%%%%%%%%%%%%%%%%%%%%%%%%%%%%%%%%%%%%%%%%%%%%%%%%%%%%%%%%%%%%%%%%%%%%%%%%%%%%%%%%%%%%%%%%%%%%%%%%%%%%%%%%%%%%%%%%%%%%%%%%%%
%(cf _doc/TL-user.pdf pour plus de d�tails)
%-------------------------------------------------------------------
%                             En-tetes
%-------------------------------------------------------------------

% Les en-tetes: quelques exemples
%\UppercaseHeadings 
%\UnderlineHeadings
%\newcommand\bfheadings[1]{{\bf #1}}
%\FormatHeadingsWith{\bfheadings}
%\FormatHeadingsWith{\uppercase}
%\FormatHeadingsWith{\underline}
\newcommand\upun[1]{\uppercase{\underline{\underline{#1}}}}
\FormatHeadingsWith\upun

\newcommand\itheadings[1]{\textit{#1}}
\FormatHeadingsWith{\itheadings}

% pour avoir un trait sous l'en-tete:
\setlength{\HeadRuleWidth}{0.4pt}

%-------------------------------------------------------------------
%                         Les references
%-------------------------------------------------------------------

\NoChapterNumberInRef
\NoChapterPrefix

%-------------------------------------------------------------------
%                           Brouillons
%-------------------------------------------------------------------

% ceci ajoute une marque `brouillon' et la date
%\ThesisDraft



%%%%%%%%%%%%%%%%%%%%%%%%%%%%%%%%%%%%%%%%%%%%%%%%%%%%%%%%%%%%%%%%%%%%%%%%%%%%%%%%%%%%%%%%%%%%%%%%%%%%%%%%%%%%%%%%%%%%%%%%%%%%%%
%%%%%TOUT CE QUI NE CONCERNE PAS THLORIA%%%%%%%%%%%%%%%%%%%%%%%%%%%%%%%%%%%%%%%%%%%%%%%%%%%%%%%%%%%%%%%%%%%%%
%%%%%%%%%%%%%%%%%%%%%%%%%%%%%%%%%%%%%%%%%%%%%%%%%%%%%%%%%%%%%%%%%%%%%%%%%%%%%%%%%%%%%%%%%%%%%%%%%%%%%%%%%%%%%%%%%%%%%%%%%%%%%%

% Solution � la hache pour �viter les warning \oval et compagnie
\makeatletter
\renewcommand\@picture@warn{}
\makeatletter


% Un nouveau compteur pour les annexes
\newcounter{compteurannexes}
\renewcommand\thecompteurannexes{\Alph{compteurannexes}}


%%%% debut macro %%%% pour refaire les t�tes de chapitres
\makeatletter
\def\@makechapterhead#1{%
  %\vspace*{50\p@}% pour enlever l'espace avant le titre => minitoc rentre dans 1 page!
  \hrule
  \vspace*{10\p@}%
  {\parindent \z@ \raggedright \normalfont
    \ifnum \c@secnumdepth >\m@ne
      \if@mainmatter
        \huge{\fontfamily{ppl}\selectfont \bfseries \@chapapp\space \thechapter}
        \par\nobreak
          \vskip 10\p@
      \fi
    \fi
    \interlinepenalty\@M
    \Huge {\fontfamily{ppl}\selectfont #1}\par\nobreak
    \vspace*{10\p@}%
        \hrule
    \vskip 10\p@
}}
%%%%%Starred chapters (toc, lof, biblio)
\def\@schapter#1{\if@twocolumn
                   \@topnewpage[\@makeschapterhead{#1}]%
                 \else
                   \@makeschapterhead{#1}%
                   \@afterheading
                 \fi}
\def\@makeschapterhead#1{%
  \hrule
  \vspace*{10\p@}%
  {\parindent \z@ \raggedright \normalfont
    \interlinepenalty\@M
    \Huge{\fontfamily{ppl}\selectfont #1}
    \par\nobreak
    \vspace*{10\p@}%
        \hrule
    %\vskip 10\p@
  }}
\rmfamily
\makeatother
%%%% fin macro %%%%


% pour changer la police de toutes les sections
\allsectionsfont{\fontfamily{ppl}\selectfont}

% Num�rote les subsubsection pour pouvoir leur mettre un label et donc les voir appara�tre dans les minitocs
\setcounter{secnumdepth}{4}

% Mise en page des subsubsection avec un bullet, �crit en sffamily
\makeatletter
\renewcommand\subsubsection{\@startsection{subsubsection}{3}{10pt}%
                                     {-1.5ex \@plus -.2ex \@minus -.2ex}%
                                     {0.1ex \@plus.2ex}%
                                     {\sffamily\bf}}
\renewcommand\thesubsubsection{$\bullet$}          
\renewcommand*\l@subsubsection{\@dottedtocline{3}{7.0em}{1.5em}}                           
\makeatother              

% profondeur de la toc principale : 1 (minitoc : 2)
\setcounter{tocdepth}{1}


%%Pour avoir le titre du chapitre dans la liste des figures
%(xchapter est g�n�r� par minitoc)
\makeatletter
\let\l@xchapter\l@chapter
\makeatother
%%%%%%%%%%%%%%%%%%%%%%%
%TITRE DES CHAPITRES
\newcommand{\ChIntro}{Introduction g�n�rale.}
\newcommand{\ChMaxent}{Mod�les de fixation des \TFs � l'ADN.}
\newcommand{\ChImogene}{{\em Imogene}: un algorithme d'identification de motifs et de modules de r�gulation transcriptionnelle}
\newcommand{\ChTrichomes}{�tude de la diff�renciation �pidermale chez la drosophile}
\newcommand{\ChMuscle}{�tude de la diff�renciation musculaire chez la souris}
\newcommand{\ChTest}{Chapitre d'exemples}


%%%%%%%%%%%%%%%%%%%%%%%%%%%%%%%%%%%%%%%%%%%%%%%%%%%%%%%%%%%%%%
%%%%%%%%% Le texte%%%%%%%%%%%%%%%%%%%%%%%%%%%
%%%%%%%%%%%%%%%%%%%%%%%%%%%%%%%%%%%%%%%%%%%%%%%%%%%%%%%%%%%%%%
\newcommand{\tf}{facteur de transcription\xspace}
\newcommand{\tfs}{facteurs de transcription\xspace}
\newcommand{\tfbs}{site de fixation de \tf\xspace}
\newcommand{\tfbss}{sites de fixation de \tf\xspace}
\newcommand{\tfbsss}{sites de fixation de \tfs\xspace}
\newcommand{\TF}{Facteur de Transcription\xspace}
\newcommand{\TFs}{Facteurs de Transcription\xspace}
\newcommand{\FT}{\TF}
\newcommand{\FTs}{\TFs}
\newcommand{\DI}{Information Directe\xspace}
\newcommand{\ID}{\DI}
\newcommand{\ft}{\tf}
\newcommand{\fts}{\tfs}
\newcommand{\TSS}{site de d�but de transcription}
\newcommand{\TSSs}{sites de d�but de transcription}
\newcommand{\CRM}{module de cis-r�gulation}
\newcommand{\CRMs}{modules de cis-r�gulation}

\newcommand{\apriori}{\textit{a priori}\xspace}
\newcommand{\aposteriori}{\textit{a posteriori}\xspace}
\newcommand{\cad}{c'est-�-dire\xspace}
\newcommand{\etal}{\textit{et al.}\xspace}

\newcommand{\denovo}{\textit{de novo}\xspace}
\newcommand{\via}{\textit{via}\xspace}
\newcommand{\insitu}{\textit{in situ}\xspace}
\newcommand{\invitro}{\textit{in vitro}\xspace}
\newcommand{\invivo}{\textit{in vivo}\xspace}
\newcommand{\chip}{ChIP\xspace}
\newcommand{\chipchip}{ChIP-on-chip\xspace}
\newcommand{\chipseq}{ChIP-seq\xspace}
\newcommand{\dmel}{\textit{Drosophila Melanogaster}\xspace}
\newcommand{\droso}{Drosophile\xspace}
\newcommand{\musmus}{\textit{Mus Musculus}\xspace}
\newcommand{\celegans}{\textit{Caenorhabditis Elegans}\xspace}
\newcommand{\ecoli}{\textit{Escherichia coli}\xspace}

\newcommand{\twi}{\textit{Twist}\xspace}
\newcommand{\myog}{\textit{Myog}\xspace}
\newcommand{\myod}{\textit{Myod1}\xspace}
\newcommand{\myf}{\textit{Myf5}\xspace}
\newcommand{\mrf}{\textit{Mrf4}\xspace}
\newcommand{\six}{\textit{Six}\xspace}
\newcommand{\six1}{\textit{Six1}\xspace}
\newcommand{\six14}{\textit{Six1,4}\xspace}
\newcommand{\six4}{\textit{Six4}\xspace}


%%%%%%%%%%%%%%%%%%%%%%%%%%%%%%%%%%%%%%%%%%%%%%%%%%%%%%%%%%%%%%
%%%%%%%%% Les maths%%%%%%%%%%%%%%%%%%%%%%%%%%%
%%%%%%%%%%%%%%%%%%%%%%%%%%%%%%%%%%%%%%%%%%%%%%%%%%%%%%%%%%%%%%

\newcommand{\Z}{\mathcal{Z}}

\newcommand{\pointformule}{\: .}
\newcommand{\virguleformule}{\: ,}
\newcommand{\finformule}{\vspace{-1cm}}

\newcommand{\OS}{\ensuremath{5^2 {\rm S}_{{1}/{2}}}\xspace}
\newcommand{\OPDD}{\ensuremath{5^2 {\rm P}_{{3}/{2}}}\xspace}
\newcommand{\OPDU}{\ensuremath{5^2 {\rm P}_{{1}/{2}}}\xspace}

\newcommand{\Etat}[1]{{| {#1} \rangle}}
\newcommand{\ket}[1]{\ensuremath{\left|#1\right\rangle}}
\newcommand{\bra}[1]{\left\langle #1\right|}


\newcommand{\Spin}[3]{\bra{#1} \widehat{S}_{#2} \ket{#3}}
\newcommand{\vr}{\ve{r}}
\newcommand{\rp}{\ve{r}^\prime}
\newcommand{\rpp}{\ve{r_\parallel}^{\prime\prime}}
\newcommand{\rA}{\ve{r_A}}
\newcommand{\uchap}{\widehat{u}}
\newcommand{\rrp}{(\ve{r},\rp)}
\newcommand{\rrpp}{(\ve{r},\rpp)}
\newcommand{\vu}{\ve{u}}

\newcommand{\FmF}[2]{{F=#1 , m_F=#2}\xspace}
\newcommand{\EtatFmF}[2]{\ensuremath{\Etat{ \FmF{#1}{#2} }}\xspace}
\newcommand{\EtatFprimemFprime}[2]{\ensuremath{\Etat{ F^\prime=#1 , m_F^\prime=#2 }}\xspace}
\newcommand{\EtatF}[1]{\ensuremath{\Etat{ F=#1 }}\xspace}
\newcommand{\EtatFprime}[1]{\ensuremath{\Etat{ F^\prime=#1 }}\xspace}
\newcommand{\mF}[1]{\ensuremath{\Etat{ m_F=#1 }}\xspace}
\newcommand{\mFprime}[1]{\ensuremath{\Etat{ m_{F^\prime}=#1 }}\xspace}
\newcommand{\mFv}[1]{\ensuremath{\Etat{ m_F }}\xspace}

\newcommand{\trdt}{\ensuremath{\EtatF{2} \to \EtatFprime{3}}\xspace}
\newcommand{\trud}{\ensuremath{\EtatF{1} \to \EtatFprime{2}}\xspace}


\newcommand{\keket}{\ensuremath{\left|\left|I=\frac32, m_I=\frac32; S=\frac12, m_s=\frac12 \right.\right\rangle}}
\newcommand{\kekett}{\ensuremath{\left|\left|I=\frac32, m_I=\frac32; S=\frac12, m_s=-\frac12 \right.\right\rangle}}
\newcommand{\kekettt}{\ensuremath{\left|\left|I=\frac32, m_I=\frac12; S=\frac12, m_s=\frac12 \right.\right\rangle}}
\newcommand{\kekk}{\EtatFmF{2}{2}}
\newcommand{\kek}{\EtatFmF{2}{1}}


\newcommand{\lambdaDB}{{\lambda_{\rm dB}}}

\newcommand{\mRb}{{m_{\rm Rb}}}
\newcommand{\Br}{{\ve{B\left( \ve{r} \right)}}}
\newcommand{\vectB}   {{\ve{B}}}
\newcommand{\vectBr}  {{\ve{B\left( \ve{r} \right)}}}
\newcommand{\vectBxyz}{{\ve{B\left( x,y,z \right)}}}
\newcommand{\modB}{{\Module{\vectB}}}
\newcommand{\modBr}{{\Module{\Br}}}

\newcommand{\omegaHF}{{\omega_{\rm hf}}}

\newcommand{\kb}{k_{\rm B}}
\newcommand{\muB}{\mu_{\rm B}}
\newcommand{\const}{{\mbox{const}}}

\newcommand{\Dix}[1]{\cdot 10^{#1}}

\newcommand{\Expo}[1]{\mathrm{e}^{\left(#1\right)}}

\newcommand{\ve}[1]{\overrightarrow{#1}}
\newcommand{\Module}[1]{\lvert #1 \rvert}

\newcommand{\Gradient}[1]{\ve{\nabla} \left( #1 \right) }
\newcommand{\Divergence}[1]{\ve{\nabla} \cdot #1 }

\newcommand{\rotr}{\ve{\nabla}_{\vr}}
\newcommand{\rotrp}{\ve{\nabla}_{\rp}}
\newcommand{\Erw}{\ve{E}(\vr,\om)}

\newcommand{\nonum}{\thispagestyle{empty}\addtocounter{page}{-1}}
\newcommand{\Ry}{Rydberg~}
\newcommand{\e}{\varepsilon}
\newcommand{\p}{\varphi}
\newcommand{\al}{\alpha}
\newcommand{\om}{\omega}
\newcommand{\de}{\delta}
\newcommand{\Om}{\Omega}
\newcommand{\ld}{\lambda}
\newcommand{\lV}{\lambda_V}
\newcommand{\lL}{\lambda_L}
\newcommand{\lac}{\lambda_{ac}}
\newcommand{\bsig}{{\boldsymbol\sigma}}
\newcommand{\si}{\sigma_i}
\newcommand{\sj}{\sigma_j}
\newcommand{\sh}{\sinh}
\newcommand{\ch}{\cosh}
\newcommand{\be}{\begin{equation}}
\newcommand{\ee}{\end{equation}}
\newcommand{\ds}{\displaystyle}
\newcommand{\bs}{\bigskip}


%\newcommand\numberthis{\addtocounter{equation}{1}\tag{\theequation}}
%\newcommand{\eq}[1]{\begin{equation}#1\end{equation*}}
%\newcommand{\eqn}[1]{\begin{equation}#1\end{equation}}
%\newcommand{\eqal}[1]{\begin{align*} #1 \end{align*}}
%\newcommand{\eqaln}[1]{\begin{align*} \numberthis#1 \end{align*}}
%\newcommand{\eqalacc}[1]{\[\left\{\begin{aligned}#1\end{aligned}\right.\]}
%\newcommand{\eqalaccn}[1]{\[\left\{\numberthis \begin{aligned}#1\end{aligned}\right.\]}

\newcommand{\Si}{\SI} % pr�vient les fautes de frappe!

\renewcommand{\tanh}{\mathrm{th}} % tangente hyperbolique
\renewcommand{\cosh}{\mathrm{ch}} % cosinus hyperbolique
\renewcommand{\sinh}{\mathrm{sh}} % sinus hyperbolique

\newcommand{\Bb}[3]{\ensuremath{\ve{B}_\text{biais}=(#1,#2,#3)}~G}

\newcommand{\omb}{\overline{\omega}}


\newcommand{\Vr}{\overrightarrow{r}}
\newcommand{\phir}{\phi(\Vr)}


\newcommand{\ReIm}[1]{\textup{#1}\,}
\renewcommand{\Re}{\ReIm{Re}}
\renewcommand{\Im}{\ReIm{Im}} 

\newcommand{\Gr}{\overleftrightarrow{G}}
\newcommand{\Gif}{\Gamma_{i\to f}}
\newcommand{\Gift}{\widetilde{\Gamma}_{i\to f}}
\newcommand{\Giftt}{\widetilde{\widetilde{\Gamma}}_{i\to f}}
\newcommand{\trif}{\ensuremath{\ket{i}\to\ket{f}}\xspace}

\newcommand{\pr}{^\prime}
\newcommand{\prpr}{^{\prime\prime}}



% Supprime de la version compil�e ce qu'il y a en argument : utile si on veut garder une trace de ce qui ne passe pas le barrage de la relecture!
\newcommand{\EnFaitNon}[1]{}

% Pour avoir des r�f�rences du style XX page YY
\newcommand{\FullRef}[1]{\ref{#1} page \pageref{#1}}
% Idem pour les �quations (rajoute des parenth�ses)
\newcommand{\FullEqRef}[1]{\eqref{#1} page \pageref{#1}}

% Pour rajouter un enlever localement une ligne � la page courante sans modifier tout le reste : tr�s utile pour la version finale
\newcommand{\AjouteLigne}{\enlargethispage{1\baselineskip}}
\newcommand{\RetireLigne}{\enlargethispage{-1\baselineskip}}

%%%%%%%%%%%%%%%%%%%%%%%%%%%%%%%%%%%%%%%%%%%%%%%%%%%%%%%
% TIKZ : Pour pouvoir avoir des supers jolies boites avec du texte et des formules exemple:
% http://www.fauskes.net/pgftikzexamples/boxes-with-text-and-math/
%%%%%%%%%%%%%%%%%%%%%%%%%%%%%%%%%%%%%%%%%%%%%%%%%%%%%%%
%%%%%%%%%%%%%%%%%%%%%%%%%%%%%%%%%%%%%%%%%%%%%%%%%%%%%%%
% Boite de D�finition
\tikzstyle{DefinitionBox}=[draw=black, fill=black!5, very thick, rectangle, rounded corners, inner sep=10pt, inner ysep=15pt]
\tikzstyle{DefinitionTitre}=[draw=black, rectangle, rounded corners, fill=white, text=black]
\newcommand{\Definition}[1]{%
\begin{center}\begin{tikzpicture}
\node [DefinitionBox] (box){%
	\begin{minipage}{0.92\textwidth}
		{\color{black}{#1}}
	\end{minipage}};
\node[DefinitionTitre, right=10pt] at (box.north west) {\sffamily D�finition\rmfamily};
\end{tikzpicture}\end{center}
}

%%%%%%%%%%%%%%%%%%%%%%%%%%%%%%%%%%%%%%%%%%%%%%%%%%%%%%%
% Boite de Notation
\tikzstyle{NotationBox}=[draw=black, fill=black!5, very thick, rectangle, rounded corners, inner sep=10pt, inner ysep=15pt]
\tikzstyle{NotationTitre}=[draw=black, rectangle, rounded corners, fill=white, text=black]
\newcommand{\Notation}[1]{%
\begin{center}\begin{tikzpicture}
\node [NotationBox] (box){%
	\begin{minipage}{0.92\textwidth}
		{\color{black}{#1}}
	\end{minipage}};
\node[NotationTitre, right=10pt] at (box.north west) {\sffamily Notation\rmfamily};
\end{tikzpicture}\end{center}
}

%%%%%%%%%%%%%%%%%%%%%%%%%%%%%%%%%%%%%%%%%%%%%%%%%%%%%%%
% Boite personnalisable
\tikzstyle{PersoBox}=[draw=black, fill=black!5, very thick, rectangle, rounded corners, inner sep=10pt, inner ysep=15pt]
\tikzstyle{PersoTitre}=[draw=black, rectangle, rounded corners, fill=white, text=black]
\newcommand{\Perso}[3]{%
\begin{figure}[!ht]
	\begin{center}\begin{tikzpicture}
	\node [PersoBox] (box){%
		\begin{minipage}{0.85\textwidth}
			{\color{black}{\small #3}}
		\end{minipage}};
	\refstepcounter{compteurencadres}\label{#2}
	\node[PersoTitre, right=10pt] at (box.north west) {\sffamily Encadr� \thecompteurencadres : #1\rmfamily};
	\end{tikzpicture}\end{center}
\end{figure}
}

%%%%%%%%%%%%%%%%%%%%%%%%%%%%%%%%%%%%%%%%%%%%%%%%%%%%%%%
% Boite de technique (un trait)
\tikzstyle{TechBox}=[draw=white, fill=black!1, rectangle, inner sep=5pt, inner ysep=10pt]
\tikzstyle{TechTitre}=[draw=black!1,  rectangle, fill=black!1, text=black]
\tikzstyle{TechLigne}=[color=black!30, line width=0.03cm, cap=round]
\newcommand{\Tech}[1]{%
\begin{center}\begin{tikzpicture}
\node [TechBox, below right,right=20pt] (box)
{%
    \begin{minipage}{0.85\textwidth}%
        {\color{black!85}{\footnotesize #1}}%
    \end{minipage}%
};
\node[TechTitre, right=0pt] at (box.north west) {\sffamily\small D�monstration\rmfamily};
\draw[TechLigne] (box.north west)--(box.south west);
\draw[TechLigne] (box.north west)++(0,-0.2)--++(1.9,0);
\end{tikzpicture}\end{center}}


%%%%%%%%%%%%%%%%%%%%%%%%%%%%%%%%%%%%%%%%%%%%%%%%%%%%%%%
% Boite d'Application num�rique
\tikzstyle{ApplicationNumeriqueBox}=[draw=green!80, fill=green!2, very thick, rectangle, rounded corners, inner sep=10pt, inner ysep=15pt]
\tikzstyle{ApplicationNumeriqueTitre}=[draw=green!80, rectangle, rounded corners, fill=white, text=black]
\newcommand{\AN}[1]{%
\begin{center}\begin{tikzpicture}
\node [ApplicationNumeriqueBox] (box){%
	\begin{minipage}{0.95\textwidth}
		{\color{black}{#1}}
	\end{minipage}};
\node[ApplicationNumeriqueTitre, right=10pt] at (box.north west) {\sffamily Application num�rique\rmfamily};
\end{tikzpicture}\end{center}
}

%%%%%%%%%%%%%%%%%%%%%%%%%%%%%%%%%%%%%%%%%%%%%%%%%%%%%%%
% Boite de Resultat important
\tikzstyle{ResultatBox}=[draw=red, fill=red!1, very thick, rectangle, rounded corners, inner sep=10pt, inner ysep=10pt]
\tikzstyle{ResultatTitre}=[draw=red, rectangle, rounded corners, fill=red, text=black]
\newcommand{\Resultat}[1]{%
\begin{center}\begin{tikzpicture}
\node [ResultatBox] (box){%
	\begin{minipage}{0.90\textwidth}
		{\color{black}{#1}\vskip-3mm}
	\end{minipage}};
%\node[ResultatTitre, right=10pt] at (box.north west) {R�sultat};
\end{tikzpicture}\end{center}
}

%%%%%%%%%%%%%%%%%%%%%%%%%%%%%%%%%%%%%%%%%%%%%%%%%%%%%%%
% Boite d'Interpr�tation
\tikzstyle{InterpretationBox}=[draw=blue, fill=blue!9, very thick,rectangle, rounded corners, inner sep=10pt, inner ysep=15pt]
\tikzstyle{InterpretationTitre}=[draw=blue, rectangle, rounded corners, fill=white, text=black]
\newcommand{\Interpretation}[1]{%
\begin{center}\begin{tikzpicture}\node [InterpretationBox] (box){%
	\begin{minipage}{0.95\textwidth}
		{\color{black}{#1}}
	\end{minipage}};
\node[InterpretationTitre, right=10pt] at (box.north west) {\sffamily Interpr�tation\rmfamily};
\end{tikzpicture}\end{center}}

%%%%%%%%%%%%%%%%%%%%%%%%%%%%%%%%%%%%%%%%%%%%%%%%%%%%%%%

%%%%%%%%%%%%%%%%%%%%%%%%%%%%%%%%%%%%%%%%%%%%%%%%%%%%%%%
% Boite de Remarque %%deux traits
\tikzstyle{RemarqueBox}=[fill=white, rectangle, inner sep=5pt, inner ysep=5pt]
\tikzstyle{RemarqueTitre}=[draw=white, thick, fill=white, text=black]
\tikzstyle{RemarqueLigne}=[color=black, line width=0.03cm, cap=round]
\newcommand{\Remarque}[1]{%
\begin{center}\begin{tikzpicture}
\node [RemarqueBox, below right] (box)
{%
    \begin{minipage}{0.85\textwidth}%
        {\color{black}{#1}}%
    \end{minipage}%
};
\node[RemarqueTitre, above right] at (box.north west) {\sffamily Remarque\rmfamily};
\draw[RemarqueLigne] (box.north west)++(0,0.6)--(box.south west);
\draw[RemarqueLigne] (box.north west)--++(2,0);
\draw (box.south west)++(-0.1,0)--(-0.1,0.6);
%\draw[very thick] (box.north east)--(box.south east);
\end{tikzpicture}\end{center}}

%%%%%%%%%%%%%%%%%%%%%%%%%%%%%%%%%%%%%%%%%%%%%%%%%%%%%%%
% Boite de Remarquee %%un trait, un titre, �crit petit
\tikzstyle{RemarqueeBox}=[fill=blue!2, rectangle, inner sep=5pt, inner ysep=5pt]
\tikzstyle{RemarqueeTitre}=[draw=white, thick, fill=white, text=black]
\tikzstyle{RemarqueeLigne}=[color=black, line width=0.03cm, cap=round]
\newcommand{\Remarquee}[2]{%
\begin{center}\begin{tikzpicture}
\node [RemarqueeBox, below right] (box)
{%
    \begin{minipage}{0.90\textwidth}%
        {\small \color{black}{#2}\normalsize}%
    \end{minipage}%
};
\node[RemarqueeTitre, above right] at (box.north west) {\sffamily Remarque~:~#1\rmfamily};
\draw[RemarqueeLigne] (box.north west)++(0,0.6)--(box.south west);
\draw[RemarqueeLigne] (box.north west)--++(2,0);
%\draw (box.south west)++(-0.1,0)--(-0.1,0.6);
%\draw[very thick] (box.north east)--(box.south east);
\end{tikzpicture}\end{center}}

%%%%%%%%%%%%%%%%%%%%%%%%%%%%%%%%%%%%%%%%%%%%%%%%%%%%%%%%%%%


%%%%%%%%%%%%%%%%%%%%%%%%%%%%%%%%%%%%%%%%%%%%%%%%%%%%%%%%%%%%%%
%%%%%%%%% Les figures%%%%%%%%%%%%%%%%%%%%%%%%%%%
%%%%%%%%%%%%%%%%%%%%%%%%%%%%%%%%%%%%%%%%%%%%%%%%%%%%%%%%%%%%%%
\newenvironment{fminipage}{%
\begin{Sbox}\begin{minipage}}
{%
\end{minipage}\end{Sbox}
\ifthenelse{\CadresOvales > 0}{\Ovalbox{\TheSbox}}{\TheSbox}
}

\newenvironment{cminipage}{%
\begin{Sbox}\begin{minipage}}
{%
\end{minipage}\end{Sbox}
\ifthenelse{\CadresOvales > 0}{\shadowbox{\TheSbox}}{\TheSbox}
}

\newcommand{\MinipageWidth}{15cm}
\newcommand{\CaptionWidth}{14.6cm}
\newcommand{\CaptionMargin}{0.4cm}
\newcommand{\FigWidth}{14.5cm}
\newcommand{\CaptionFig}[1]{\caption{\small{#1}}}
\newcommand{\CaptionTocCaptionFig}[2]{\caption[#1]{\small{#2}}}
\newcommand{\captionbf}[2]{\caption[#1]{\small \textbf{#1.}{#2}}}

% Les figures ombr�es *********************************************
\newenvironment{GaelFigureSh}[1][\MinipageWidth]{%
\begin{figure}[!ht]
\centering
\begin{cminipage}{#1}
\centering
\vspace{\CaptionMargin}
\pgfmathqparse{#1-\CaptionMargin-\CaptionMargin}
\begin{minipage}{\pgfmathresult pt}
\centering
}{%
\end{minipage}
\end{cminipage}
\end{figure}}
\newcommand{\bfigsh}{\begin{GaelFigureSh}}
\newcommand{\efigsh}{\end{GaelFigureSh}}

% Les figures normales *********************************************
\newenvironment{GaelFigure}[1][\MinipageWidth]{%
\begin{figure}[!ht]
\centering
\begin{fminipage}{#1}
\centering
\vspace{\CaptionMargin}
\pgfmathqparse{#1-\CaptionMargin-\CaptionMargin}
\begin{minipage}{\pgfmathresult pt}
\centering
}{%
\end{minipage}
\end{fminipage}
\end{figure}}
\newcommand{\bfig}{\begin{GaelFigure}}
\newcommand{\efig}{\end{GaelFigure}}

% Les figures : pleine page ***************************************
\newenvironment{GaelFigurep}[1][\MinipageWidth]{%
\begin{figure}[!htbp]
\centering
\begin{fminipage}{#1}
\centering
\vspace{\CaptionMargin}
\pgfmathqparse{#1-\CaptionMargin-\CaptionMargin}
\begin{minipage}{\pgfmathresult pt}
\centering
}{%
\end{minipage}
\end{fminipage}
\end{figure}}
\newcommand{\bfigp}{\begin{GaelFigurep}}
\newcommand{\efigp}{\end{GaelFigurep}}


%%%%%%%%%%%%%%%%%%%%%%%%%%%%%%%%%%%%%%%%%%%%%%%%%%%%%%%%%%%%%%
%%%%%%%%% Les tableaux%%%%%%%%%%%%%%%%%%%%%%%%%%%
%%%%%%%%%%%%%%%%%%%%%%%%%%%%%%%%%%%%%%%%%%%%%%%%%%%%%%%%%%%%%%
\newenvironment{CedTableau}[1][\MinipageWidth]{%
\begin{table}[!ht]
\centering
\begin{fminipage}{#1}
\centering
\vspace{\CaptionMargin}
\pgfmathqparse{#1-\CaptionMargin-\CaptionMargin}
\begin{minipage}{\pgfmathresult pt}
\centering
}{%
\end{minipage}
\end{fminipage}
\end{table}}
\newcommand{\btab}{\begin{CedTableau}}
\newcommand{\etab}{\end{CedTableau}}


%%%%%%%%%%%%%%%%%%%%%%%%%%%%%%%%%%%%%%%%%%%%%%%%%%%%
%% Pour faire des supers fleches d'highlight entre texte et �quation %%
%% http://www.fauskes.net/pgftikzexamples/global-nodes/
%% On d�finit un bout de fleche, et un d�part de fleche, puis on dit de faire les fleches....

\newcommand{\DepartHighlightColorNoTexte}[1]{ \tikz\node[fill=#1!20,draw, circle](From#1){};}

\newcommand{\DepartHighlightColor}[2]{ \tikz[baseline=(From#1.base)] {\node[fill=#1!20](From#1){#2};}}

\newcommand{\ArriveeHiglightColor}[2]{
\tikz[baseline=(To#1.base)] {\node[fill=#1!20](To#1){#2};}}

\newcommand{\MakeFlecheHighlightColor}[1]{ \begin{tikzpicture}
[overlay]\path[->](From#1)edge%[bend left]
(To#1);
\end{tikzpicture}}


%%%%%%%%%%%%%%%%%%%%%%%%%%%%%%%%%%%%%%%%%%%%%%%%%%%%%%%%%%%%%%
%%%%%%%%% Les listes%%%%%%%%%%%%%%%%%%%%%%%%%%%
%%%%%%%%%%%%%%%%%%%%%%%%%%%%%%%%%%%%%%%%%%%%%%%%%%%%%%%%%%%%%%
%check tick : \surd
%fl�che droite creuse : \triangleright
%rond : bullet 	
%%%REFERENCE : http://www.artofproblemsolving.com/LaTeX/AoPS_L_GuideSym.php


\newenvironment{maliste}[1]%
{ \begin{list}%
	{$\triangleright$}%
	{\setlength{\labelwidth}{10pt}%
	 \setlength{\leftmargin}{#1pt}%
	 %\setlength{\itemsep}{\parsep}
	\setlength{\itemsep}{0pt}
}}%
{ \end{list} }

\newenvironment{malistee}[1]%
{ \begin{list}%
	{$\to$}%
	{\setlength{\labelwidth}{0pt}%
	 \setlength{\leftmargin}{#1pt}%
	 %\setlength{\itemsep}{\parsep}
	\setlength{\itemsep}{0pt}
}}%
{ \end{list} }

\newenvironment{malistebullet}[1]%
{ \begin{list}%
	{$\bullet$\refstepcounter{subsubsection}}%
	{\setlength{\labelwidth}{0pt}%
	 \setlength{\leftmargin}{#1pt}%
	 %\setlength{\itemsep}{\parsep}
	%\setlength{\itemsep}{0pt}
}}%
{ \end{list} }

\newcommand{\blst}[1]{\begin{maliste}{#1}}
\newcommand{\elst}{\end{maliste}}
\newcommand{\blste}[1]{\begin{malistee}{#1}}
\newcommand{\elste}{\end{malistee}}
\newcommand{\blstbul}[1]{\begin{malistebullet}{#1}}
\newcommand{\elstbul}{\end{malistebullet}}

%%Liste personnalisable
\newenvironment{malisteperso}[2]%
{ \begin{list}%
	{$#2$}%
	{\setlength{\labelwidth}{10pt}%
	 \setlength{\leftmargin}{#1pt}%
	 %\setlength{\itemsep}{\parsep}
	\setlength{\itemsep}{0pt}
}}%
{ \end{list} }


%%%%%%%%%%%%%%%%%%%%%%%%%%%%%%%%%%%%%%%%%%%%%%%%%%%%%%%%%%%%%%
%%%%%%%%% Les annexes (c'est crad de mettre �a l� mais bon!) %
%%%%%%%%%%%%%%%%%%%%%%%%%%%%%%%%%%%%%%%%%%%%%%%%%%%%%%%%%%%%%%

\newcommand{\AnnexTitreCed}[4]
{
%\pagestyle{plain}
\renewcommand{\leftmark}{Annexe #1 - #2}
\renewcommand{\rightmark}{Annexe #1 - #2}
 %\hrule
  \vspace{15pt}%
  {\Huge \fontfamily{ppl}\selectfont \textbf{Annexe #1}}
   \par\nobreak
    \vspace{20pt}
   {\Huge \fontfamily{ppl}\selectfont #2}
    \vspace{10pt}
       \hrule
   \vskip20pt
	\refstepcounter{compteurannexes}\label{#3}
  \normalsize
  #4
  \EmptyNewPage
}


\newcommand{\Article}[1]
{
%\pagestyle{plain}
\renewcommand{\leftmark}{Article}
\renewcommand{\rightmark}{Article}
 \hrule
  \vspace{15pt}%
   {\Huge \fontfamily{ppl}\selectfont Article}
    \vspace{10pt}
       \hrule
   \vskip20pt
  \normalsize
  #1
  \EmptyNewPage
}



%%%%%%%%%%%%%%%%%%%%%%%%%%%%%%%%%%%%%%%%%%%%%%%%%%%%%%%%%%%%%%
%%%%%%%%% Pour la liste des notations %%%%%%%%%%%%%%%%%%%%%
%%%%%%%%%%%%%%%%%%%%%%%%%%%%%%%%%%%%%%%%%%%%%%%%%%%%%%%%%%%%%%


	% ************************************************************************
		% Pour fair une note dans la marge !
		% Command for inserting a todo item
		\definecolor{orange}{rgb}{1,0.5,0}
		\tikzstyle{notestyleright} = [right, draw=black, fill=blue!1]%, text width = 1cm]
		\tikzstyle{notestyleleft} = [notestyleright, left]
		\tikzstyle{connectstyle} = [draw = orange, thick]
		\newcommand{\todo}[1]{%
		% Add to todo list
		%\addcontentsline{tdo}{todo}{\protect{#1}}%
		%
		\begin{tikzpicture}[remember picture]%, baseline=7.5ex]%
		  \node [coordinate] (inText) {};
		\end{tikzpicture}%
		%
		% Make the margin par
		\marginpar[%
		{% Draw note in left margin
		 \tikz[remember picture] \path (0,0)++(1.8,0)++(-0.2,0)node[notestyleleft] (inNote) {\hyperlink{nomenclaturepage}{#1}};}%
		]{% Draw note in right margin
		  \tikz[remember picture] \path (0,0)++(0.2,0)node[notestyleright] (inNote) {\hyperlink{nomenclaturepage}{#1}};%
		}
		}

 %\ChCourantTitre est d�fini dans le fichier packages...
	\newcommand{\nome}[2]{%
      \nomenclature[\ChCourantTitre]{#1}{#2}%
      \todo{#1}%  ---------------------------------------la commande todo{truc} fout le truc dans la marge du bon cot�
  }

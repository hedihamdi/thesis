%%%%%%%%%%%%%%%%%%%%%%%%%%%%%%%%%%%%%%%%%%%%%%%%%%%%%%%%%
%%%% PACKAGES FONDAMENTAUX
%%%%%%%%%%%%%%%%%%%%%%%%%%%%%%%%%%%%%%%%%%%%%%%%%
\usepackage{ifthen}
\usepackage{ifpdf}


\usepackage[pdftex]{graphicx} % pour l'incusion de graphiques en pdf

\usepackage[intlimits]{amsmath} % amsmath, pour les maths
\usepackage{amssymb} % amssymb pour les symboles maths

%Pour pouvoir utiliser \allsectionsfont (utilis� dans le template)
\usepackage{sectsty}

% C'est quand m�me mieux, un pdf cliquable!
\usepackage[pdftex, colorlinks = true
%\usepackage[pdftex, colorlinks = false
   , pdfstartview = FitH 
   , linkcolor = blue
   , citecolor = blue
   , urlcolor = blue
   , bookmarksnumbered = true
   , bookmarksopen = true
   , pdfpagelabels
, pagebackref = true ]{tlhypref}


\usepackage{natbib} % citation of type (Dupont et al., 2013)
\renewcommand{\cite}{\citep}

\usepackage{morefloats} % accomodate large number of floats
\usepackage{placeins} 

\usepackage{fancybox} %pour encadrer les images en oval, shadow... Utilis� dans les macros

\usepackage{appendix} % pour annexes
\renewcommand\appendixname{Annexe}
\renewcommand\appendixpagename{Annexes}

%\usepackage{setspace} % pour singlespacing pour tabular



%%% Backref : permet de faire automatiquement des r�f�rences en arri�re cliquables dans la biblio, tr�s utile!
% setup the style of the backrefs from the bibliography %version ga�l compliqu�!
\RequirePackage[hyperpageref]{backref} % to be loaded after hyperref package
\newcommand{\backrefnotcitedstring}{\relax}%(Not cited.)
\newcommand{\backrefcitedsinglestring}[1]{(Page~#1.)}
\newcommand{\backrefcitedmultistring}[1]{(Pages~#1.)}
\renewcommand{\backreftwosep}{ et~} % seperate 2 pages
\renewcommand{\backreflastsep}{, et~} % seperate last of longer list
\renewcommand*{\backref}[1]{}  % Disable standard
\renewcommand*{\backrefalt}[4]{% Detailed backref
   \ifcase #1 %
   \backrefnotcitedstring%
   \or
   \backrefcitedsinglestring{#2}%
   \else
   \backrefcitedmultistring{#2}%
\fi}

% Si on veut des mini-tables des matieres :
\ifthenelse{\FaitMinitocs > 0}{
   \usepackage[french,nohints]{minitoc}%
   \mtcsetrules{minitoc}{off} % pour virer les lignes horizontales
   \mtcsettitle{minitoc}{} % pour virer le mot "<Sommaire"> au d�but des minitoc
   \setlength{\mtcindent}{0pt} % moins d'indentation � gauche des minitoc (d�faut 24pt)
   \mtcsetdepth{minitoc}{2} % pour voir sections et subsections.
}{}


%%%%%%%%%%%%%%%%%%%%%%%%%%%%%%%%%%%%%%%%%%%%%%%%%%%%%%%
% Pour pouvoir avoir des fleches depuis le texte vers des formules par exemple: TIKZ. Sert aussi pour l'index des notations, et pour les cadres
% http://www.fauskes.net/pgftikzexamples/global-nodes/
%%%%%%%%%%%%%%%%%%%%%%%%%%%%%%%%%%%%%%%%%%%%%%%%%%%%%%%
\usepackage{tikz,tkz-fct,tkz-2d}			% pour des supers graphiques trop cool, mieux que PSTricks.
\usetikzlibrary{snakes,arrows,shapes,backgrounds}
% For every picture that defines or uses external nodes, you'll have to apply the 'remember picture' style. To avoid some typing, we'll apply the style to all pictures.
\tikzstyle{every picture}+=[remember picture] 

% espace intelligent pour la fin d'une newcommand en tex (utilis� partout dans le fichier macro pour les maths)
\usepackage{xspace}

%gestion des unit�s !
\usepackage{sistyle}
%\SIstyle{German}   % Utile pour avoir une virgule avant les d�cimales et pas un point

% Les polices latex
\usepackage[latin1]{inputenc}
\usepackage[T1]{fontenc}
\usepackage{lmodern}
\usepackage[frenchb]{babel}

\usepackage[osf]{mathpazo}   % Palatino with smallcaps and oldstyle numbers
\usepackage[scaled]{helvet}  % Helvetica, scaled 95%


%Pour faire l'index des notations
\ifthenelse{\FaitNomenclature > 0}{
    \usepackage[norefpage,french,intoc]{nomencl}  %intoc rajoute une ligne dans la toc
    \renewcommand{\nomname}{\fontfamily{ppl}\selectfont Principales abbr�viations utilis�es} %Titre de la page
    \renewcommand{\pagedeclaration}[1]{~\dotfill\hyperpage{#1}}  %pour avoir des lignes de ... entre la description et le #page
    \renewcommand{\nomgroup}[1]{\bigskip}
    %\renewcommand{\nomgroup}[1]{%
    %\ifthenelse{\equal{#1}{A}}{\item[\textbf{\ChIntro}]}	  {
        %\ifthenelse{\equal{#1}{B}}{\medskip\item[\textbf{\ChMaxent}]}{
            %\ifthenelse{\equal{#1}{C}}{\medskip\item[\textbf{\ChImogene}]}{
                %\ifthenelse{\equal{#1}{D}}{\medskip\item[\textbf{\ChTrichomes}]}{
                    %\ifthenelse{\equal{#1}{E}}{\medskip\item[\textbf{\ChMuscle}]}{
                        %}}}}}
                    %}
                    %\def\ChCourantTitre{\ifcase\value{chapter} \or A\or B\or C \or D \or E\fi}
   %%%%%%%%%%%%%%%%%%%%%%%%%%%% D�finition de \ChCourantTitre ici!!!	


            \makenomenclature
         }{\newcommand{\nome}[2]{}}



% Pour ajouter des lignes comme biblio � la table des mati�res. Utilis� dans le template
         \usepackage[none]{tocbibind} 

%%%%%%%%%%%%%%%%%%%%%%%%%%%%%%%%%%%%%%%%%%%%%%%%%%%%%%%%%�
%%%% PACKAGES NON FONDAMENTAUX
%%%%%%%%%%%%%%%%%%%%%%%%%%%%%%%%%%%%%%%%%%%%%%%%%

%permet d'inclure des fichiers pdf directement dans le tex (articles en annexe), avec \includepdf[pages=-]{votre_fichier} 
         \usepackage{pdfpages}


         \usepackage{picins} %pour \parpic : images dans le texte � gauche ou � droite...

%http://www.cs.brown.edu/system/software/latex/doc/eqnarray.pdf
%Le fichier est � la racine du projet.
         \usepackage{equationarray}

% pour avoir les symboles \diameter, \frownie et \smiley
         \usepackage{wasysym}

% pour avoir les \ding{} : num�ros encercl�s
         \usepackage{pifont}

% pour des repr�sentation de vecteur sympa pour les physiciens
         \usepackage{vector}

% figures avec sous-figures et sous-captions:
         \usepackage[labelformat=empty, caption=true]{subfig}

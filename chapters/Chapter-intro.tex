\vskip7mm
aller rapidement sur nouvelles techniques.
statistiques du genome

\noindent\textbf{\LARGE\sffamily\scshape Diff�rentes cellules expriment
diff�rent g�nes}
\section{Diff�rentes cellules expriment diff�rent g�nes}
\medskip


\begin{malistebullet}{15}
\bs\item{\textbf{\sffamily{Les cellules se sp�cialisent au cours du
   d�veloppement}}}\\
   Waddington, destin cellulaire. 
\bs\item{\textbf{\sffamily{Les cellules sont reprogrammables}}}\\
   Fibroblastes, IPS : seulement un ou quelques facteurs suffisent � changer le
   ph�notype d'une cellule.
\end{malistebullet}

$\rightarrow$ comment interpr�ter ces r�sultats?

\vskip7mm
\noindent\textbf{\LARGE\sffamily\scshape Les r�seaux de r�gulation
g�n�tique }
\medskip

\begin{malistebullet}{15}
\bs\item{\textbf{\sffamily{Bref historique}}}\\
   Monod, Jacob. Promoteurs. 
\bs\item{\textbf{\sffamily{Divers modes de r�gulation}}}\\
   Enhancers, �pig�n�tique, post-transcriptionnelle, etc.
\end{malistebullet}

$\rightarrow$ quels outils pour exhiber cette circuiterie?

%%%%%%%%%%%%%%%%%%%%%%%%%%%%%%%%%%%%%%%%%%%%%%%%%%%%%%%%%%%%%%%%%%%%%%%%%%%%%%%%%%%%%%%%%%%%%%%%%%%%
\vskip7mm
\noindent\textbf{\LARGE\sffamily\scshape D�crire les interactions au seins des
r�seaux.}
\medskip

\begin{malistebullet}{15}
\item R�gulation transcriptionnelle. Facteurs de transcription. Diffusion, fixation. \\
\item   Mod�les math�matiques (PWM, biophysiques). \\
\item   Coop�rativit�, fonctions logiques, coefficients de Hill (Uri Alon).\\
\item   Donn�es biologiques grande �chelle : \chipseq etc\ldots Bioinformatique.\\
\item   ENCODE, Taipale, etc.\\
\item   Evolution de la r�gulation Odom, Sinha.\\
\end{malistebullet}


Visualisation sur UCSC.


%\newpage \noindent\textbf{\sffamily\scshape Quelques remarques sur la version
%pdf du manuscrit}

Voici quelques remarques sur la version pdf de ce manuscrit, qui peuvent rendre
la lecture plus ais�e. Dans la table des mati�res, la liste des figures et la
liste des annexes, les titres sont des liens hypertexte qui pointent vers
l'item d�crit. Dans la liste des notations utilis�es et la bibliographie, ce
sont les num�ros de page qui sont des liens hypertexte.

